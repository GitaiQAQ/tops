\chapter{Sequence Formats}


ToPS can read two distinct text-based file format: (1) FASTA; (2) ToPS sequence format.  

\section {FASTA format}

FASTA format has become a standard in the field of Bioinformatics and it represents any sequence data such as  nucleotide sequences or protein sequences.

A sequence in FASTA format always begins with a single-line description, followed by the lines of the sequence data.  The description line begins with the greater-than (">") symbol. The sequence data ends when another ">" appears or when the end-of-file is encountered. 

\vspace{1em}

\noindent {\bf Example:}

\begin{Verbatim}
>chr13_72254614
AGGAGAGAATTTGTCTTTGAATTCACTTTTTCTTACCTATTTCCC
TTCAAAAAGAAGGAGGGAGGCCGATCTTAGTATAGTTCTCGCTGT
TTCCCCTCCACACACCTTTCCTTATCATTCAGTTTAGAAAAACTG
AAATATTTAATAGCATAATTTGTTATATCATGAGGTATTAAAACA
AGGTAGTTGCTAACATTCTTATGAGAGAGTTAGAAGTAAGTTCTA
>chr12_54396566
ACTCTGGAGGGAGGAGGGTGTGGGGAACCCCCCAGAGATGGGCTT
CTTGGAGGCCTGAAACCACCGGAACGGAGGTGGGGCACTTGTTTC
CTGAGTCCGGGCTGGAAATCTCGGAGTTACCGATTCTGCGGCCGA
GTAGTGGAGAAAGAGTGCCTGGGAGTCAGGAGTCCTGGGCGCTGC
CGCTGACTTCCTGGCGTCCCTGAGTGAGTCCATTTCCCTCCCAGG
\end{Verbatim}

\section{ToPS sequence format}

ToPS sequence format assumes that each line contains a single sequence data. 
A sequence in ToPS sequence format begins with a description followed by the two-dots (":") symbol, followed by a space, followed by the sequence data. ToPS allows the presence of multiple-character symbols, and the symbols in the sequence data are isolated by a space. 

\vspace{1em}
\noindent {\bf Example:}

\begin{Verbatim}
chr13_72254614: A G G A G A G A A T T T G T C T T T
seq1234: CPG CPG NOCPG CPG CPG CPG CPG NOCPG 
\end{Verbatim}



