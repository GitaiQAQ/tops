\chapter{Evaluating probabilities of a sequence}

ToPS provides a program, called  \textit{evaluate}, which calculates the probability of sequences given a probabilistic model,  requires as parameters the model description and a set of sequences. For example, the command line below determines the probabilities of a set of sequences (sequences.txt) given the HMM model.


\begin{Verbatim}[frame=single, label={Command line}]
evalulate -m hmm.txt < sequences.txt
\end{Verbatim}

The command line parameter of the \textit{evaluate} program is:
\begin{itemize}
\item -m specifies the name of the file containing the model description.
\end{itemize}

\begin{Verbatim}[frame=single, label={hmm.txt}]
model_name="HiddenMarkovModel"
state_names= ("1", "2" )
observation_symbols= ("0", "1" )
# transition probabilities
transitions = ("1" | "1": 0.9;
                            "2" | "1": 0.1;
                            "1" | "2": 0.05;
                       "2"| "2": 0.95 )
# emission probabilities
emission_probabilities = (
                         "0" | "2" : 0.95; 
                         "1" | "2" : 0.05;            
                         "0" | "1" : 0.95; 
                         "1" | "1" : 0.05)
initial_probabilities= ("1": 0.5; "2": 0.5)
\end{Verbatim}

The evaluate program is not limited to the use with HMM, any probabilistic model description works as an input. 


